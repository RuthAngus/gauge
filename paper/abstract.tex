% Motivation
The ages of main sequence stars are extremely difficult to measure, however
they are of great interest to the fields of galactic archaeology and exoplanet
formation and evolution.
Two promising proxies for stellar age are rotation period and vertical action.
Rotation periods of MS stars decrease over time due to magnetic braking and
can be used as a clock via `gyrochronology'.
However, a lack of suitable calibrators, especially at old ages, leaves an
unsolved mystery: do stars continue to spin down for the entirety of their
time on the MS or does magnetic braking cease at some critical age?
Vertical action provides a weak estimate of stellar age, but, unlike other
age-dating methods, age-independent one.
Using vertical action as an age proxy therefore allows us to investigate the
spin-down behaviour of old stars.
% Goal
We use \Gaia\ postions, proper motions and parallaxes and \Kepler\ rotation
periods of a hundred stars to establish the relationship between vertical
action-age and rotation-age of MS dwarfs.
% Method & data
% Results
We demonstrate that gyrochronology age increases with vertical action
dispersion.
% Interpretation
This allows establish 1) that vertical action can be used as an age
proxy for MS dwarfs and 2) that stellar age increases with height from the
galactic plane.
Further, we show that the relation between vertical action and rotation-age
does not strongly depend on the specific gyrochronology model used to infer
age.

% We calculate the ages of the \nstar\ stars with previously published
%     rotation periods that also feature in the TGAS (Tycho-Gaia Astrometric
%     Solution) database using a simple gyrochronology relation.
% We also estimate the vertical actions of these stars from their
%     two-dimensional TGAS proper motions and parallax.
% Using radial velocity measurements obtained for a subset of these stars, we
%     demonstrate that the vertical actions calculated from the two-dimensional
%     proper motions are adequately estimated due to the specific orientation of
%     the \Kepler\ field.
% We demonstrate that the gyrochronal ages of these stars scales with vertical
%     action dispersion.
% We show that the vertical action-age relations vary with stellar mass.
% This could either reflect the different ages of the different stellar
%     populations, or could be the result of an error in the gyrochronology
%     relations.
% Radial velocities of these stars, to be published in the second \gaia\ data
%     release should further clarify the relationship between gyrochronology age
%     and vertical action.

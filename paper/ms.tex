\documentclass[useAMS, usenatbib, preprint, 12pt]{aastex}
\usepackage{cite, natbib}
\usepackage{float}
\usepackage{epsfig}
\usepackage{cases}
\usepackage[section]{placeins}
\usepackage{graphicx, subfigure}
\usepackage{color}
\usepackage{bm}

\newcommand{\columbia}{1}
\newcommand{\simons}{2}
\newcommand{\nyu}{3}
\newcommand{\cca}{4}
\newcommand{\mpia}{5}
\newcommand{\cds}{6}
\newcommand{\naigrain}{333}
\newcommand{\nmcquillan}{100}
\newcommand{\kepexample}{1430163}
\newcommand{\kepexampleperiod}{4}
\newcommand{\aigrainexampleperiod}{20.8}
\newcommand{\Kepler}{{\it Kepler}}
\newcommand{\kepler}{\Kepler}
\newcommand{\corot}{{\it CoRoT}}
\newcommand{\Ktwo}{{\it K2}}
\newcommand{\ktwo}{\Ktwo}
\newcommand{\TESS}{{\it TESS}}
\newcommand{\LSST}{{\it LSST}}
\newcommand{\Wfirst}{{\it Wfirst}}
\newcommand{\SDSS}{{\it SDSS}}
\newcommand{\PLATO}{{\it PLATO}}
\newcommand{\Gaia}{{\it Gaia}}
\newcommand{\gaia}{{\it Gaia}}
\newcommand{\Teff}{$T_{\mathrm{eff}}$}
\newcommand{\teff}{$T_{\mathrm{eff}}$}
\newcommand{\FeH}{[Fe/H]}
\newcommand{\feh}{[Fe/H]}
\newcommand{\ie}{{\it i.e.}}
\newcommand{\eg}{{\it e.g.}}
\newcommand{\logg}{log \emph{g}}
\newcommand{\dnu}{$\Delta \nu$}
\newcommand{\numax}{$\nu_{\mathrm{max}}$}

\newcommand{\nstar}{\racomment{CH4NG3M3!}}

\newcommand{\racomment}[1]{{\color{red}#1}}

\begin{document}

\title{Measuring stellar ages with vertical kinematics}
% Testing gyrochronology with galactic kinematics from \Gaia

\author{%
   Ruth Angus\altaffilmark{\columbia, }\altaffilmark{\simons},
   David W. Hogg\altaffilmark{\nyu, }\altaffilmark{\cca, }\altaffilmark{\mpia}
   Melissa Ness\altaffilmark{\columbia, }\altaffilmark{\mpia}
   {\it et al.}
}

\altaffiltext{\columbia}{Department of Astronomy, Columbia
University, NY, NY}
\altaffiltext{\simons}{Simons Fellow, RuthAngus@gmail.com}
\altaffiltext{\nyu}{Center for Cosmology and Particle Physics, New York
University, NY, NY}
\altaffiltext{\cca}{Center for Computational Astrophysics, Flatiron Institute,
NY, NY}
\altaffiltext{\mpia}{Max Planck Institute of Astronomy, Heidelberg, Germany}
\altaffiltext{\cds}{Center for Data Science, New York University, NY, NY}


\begin{abstract}
Using the proper motions and positions of \nstar\ main sequence \Gaia\ and
    \Kepler\ targets with previously measured rotation periods, we demonstrate
    that gyrochronology age increases with vertical action dispersion.
We calculate the ages of the \nstar\ stars with previously published
    rotation periods that also feature in the TGAS (Tycho-Gaia Astrometric
    Solution) database using a simple gyrochronology relation.
We also estimate the vertical actions of these stars from their
    two-dimensional TGAS proper motions and parallax.
Using radial velocity measurements obtained for a subset of these stars, we
    demonstrate that the vertical actions calculated from the two-dimensional
    proper motions are adequately estimated due to the specific orientation of
    the \Kepler\ field.
We demonstrate that the gyrochronal ages of these stars scales with vertical
    action dispersion.
We show that the vertical action-age relations vary with stellar mass.
This could either reflect the different ages of the different stellar
    populations, or could be the result of an error in the gyrochronology
    relations.
Further, we show that the relation between vertical action and age does not
    strongly depend on the specific gyrochronology model used to infer age.
\racomment{Why is this?}
Radial velocities of these stars, to be published in the second \gaia\ data
    release should further clarify the relationship between gyrochronology age
    and vertical action.
\end{abstract}

\section{Introduction}

Age is one of the most elusive stellar parameters, particularly for stars on
the main sequence.
Isochrone fitting methods suffer from poor precision, especially when only
photometric information is available, and asteroseismology is limited to a
small sample of bright \kepler\ short cadence targets.
Stellar rotation is often used to infer ages via gyrochronology, however it
remains relatively untested at old ages due to the sparsity of precise data.
The data in hand tell an intriguing story: they indicate that magnetic braking
may cease altogether for old, slowly rotating stars.
Here we investigate another dating method: galactic kinematics.

Over the last few decades, observations of Milky Way stars with a range of
ages have revealed that older stars have greater velocity dispersions than
younger stars \citep[\eg][]{casagrande2011, aumer2011}.
It is hypothesized that interactions between stars and spiral arms or massive
gas clouds cause orbital heating and over time the mean velocity of a star
will, on average, increase.
Velocity however is not time invariant: it depends a star's orbital phase --
for example whether it is closer to pericenter or apocenter.
On the other hand, action angles {\it are} time invariant.
The action of a star is its angular momentum integrated over the Milky Way
potential (no integration would be necessary if the MW were a point mass at
the galactic center) and this is constant no matter what orbital phase you
happen to observe a star at.
For this reason we use the age-action dispersion relation
% Aumer M., Binney J. J., 2011, MNRAS, 397, 1286
% Casagrande L., Scho ̈nrich R., Asplund M., Cassisi S., Ram ́ırez I., Mele ́ndez
J., Bensby T.,
% Feltzing S., 2011, A&A, 530, A138
We calculate the dispersion in vertical action for Gaia DR1 targets in the
\kepler\ field.
Although expected to be a weak tracer of age, vertical action dispersion
should be an accurate one as the underlying physical processes are relatively
simple and well understood.
We test the potential of vertical action dispersion as a clock, by comparing
age predictions to asteroseismic ages.
Using the rotation periods of \kepler\ stars we also test the gyrochronology
relations: does magnetic braking cease at late ages?

The processes behind the formation of the galaxy and the formation of
exoplanets are two complex topics in astrophysics connected by a common theme:
stellar ages.
Main sequence (MS) stars comprise the majority of our galaxy but unfortunately
their ages are notoriously difficult to measure \citep[see][for a
review]{Soderblom2010}.
Their positions on the HR diagram don't change significantly during their
hydrogen burning lifetimes, a fact that is convenient for life on Earth but
inconvenient for galactic archaeologists.
Now, due to the abundance of rotation periods for MS stars provided by Kepler
and to-be provided by TESS, LSST and Wfirst, rotation-dating has the potential
to be the most readily available precise method for inferring stellar ages.
Rotation-dating works well for young stars but its accuracy is questionable
for stars older than the Sun: old \kepler\ asteroseismic stars rotate more
rapidly than expected given their age \citep{Angus2015, Vansaders2016,
Metcalfe2016}.
\citet{Vansaders2016} attribute this to an evolving magnetic dynamo: as stars
reach an critical Rossby number (the ratio of rotation period to the
convective overturn timescale), their magnetic field `switches off' and stars
maintain a constant rotation period after that time.
The data sets typically used to test the age-rotation relations are highly
heterogeneous and each set has its own detection and selection biases.
For example, asteroseismology favours quiet stars whereas rotation periods are
easiest to measure for active stars.
Here, we test the age-rotation relations by comparing rotational (gyrochronal)
age with dynamical age.
Data from the first \Gaia\ data release (DR1) provides kinematic ages for
several MS \Kepler\ stars with detectable rotation periods.
We use these kinematic ages to test the gyrochronology relations at all ages.
% We use the vertical action-age relation for red giants to test the
%     vertical action-age relation for MS stars.

% A history of rotation-dating.
\subsection{The status of gyrochronology}
The phenomenon of magnetic braking in MS stars was first observed almost fifty
years ago by \citet{Skumanich1972} who noticed that the rotation periods of
the Sun and young cluster stars decayed with the square-root of time.
% Later, a mass-dependence was added to the relation between age and rotation
% period --- less massive stars lose angular momentum faster than more massive
% one.
\citet{Kawaler1988} derived a formalism for this angular momentum loss.
 % and his
% relation depended on the mass loss rate, the ....
More recently, \citet{Barnes2003} demonstrated that a simple relation could be
used to describe `gyrochronology', the method of rotation-dating, and further
works \citep[\eg][]{Barnes2007, Mamajek2008, Barnes2010, Meibom2011},
demonstrated that the relation between rotation period and age holds true
while theorists \citep[\eg][]{Matt2012, Epstein2014, Vansaders2013,
Vansaders2015} modify and extend physical models that reproduce the
observations.

As discussed above, the performance of the age-rotation relations for old
stars was recently called into question \citep{Angus2015, Vansaders2016,
Metcalfe2016}.
A simple straight-line model for rotational age does not reproduce the ages
predicted by asteroseismology for stars older than the Sun.
This phenomenon is attributed to a transitioning magnetic dynamo at a critical
Rossby number, $Ro$, of 2.6, the solar value \citep{Vansaders2016}.
As rotation periods slow, $Ro$ decreases until it hits the critical value and
magnetic braking switches off: stars maintain their rotation period from that
point onward.
This restricts the applicability of gyrochronology to young, rapidly rotating
stars only.
This hypothesis is supported by the existing data, however these data are
sparse~---~the analyses demonstrating the discrepancies were conducted on the
small number of main sequence, Solar-like oscillators observed by Kepler in
short cadence mode with detectable rotation periods: a sample size of around
20.
% Furthermore, asteroseismology favors stars with low surface gravities since
% they show the largest amplitudes of oscillation and the majority of stars in
% this sample were slightly evolved.
A larger sample of old main sequence stars with precise ages is required to
confirm and further characterize the Rossby saturation mechanism introduced by
\citep{Vansaders2016}.
However, old main sequence stars are difficult to age-date with any other
method than asteroseismology.
We use the age gradients in the kinematic properties of stars to investigate
the \citet{Vansaders2016} model.

\subsection{Kinematic ages}
The age-velocity dispersion relations are based on relatively simple physics.
There is evidence to suggest that stars in the Milky Way form in the thin disc
of the galaxy with relatively small vertical velocities and angular momenta
\racomment{add citations}.
As time passes these stars are scattered via close encounters with other
stars and interactions with galactic spiral arms.
The more time passes, the more scattering events, and stars slowly accumulate
angular momentum in the vertical direction, known as vertical action.
This results in a slow heating of stellar orbits.
Older stars can be identified in Gaia DR1 by integrating their orbits in the
potential of the Milky Way to convert their proper motions, positions and
parallaxes into vertical actions.
It is the {\it dispersion} in vertical action that truly traces age, far
better than vertical action itself.
However for single stars only the individual's vertical action is available.
Nonetheless, $J_z^2$ increases over time for any given star, and is a weak
tracer of age.

\section{The Data}

\begin{itemize}
    \item{TGAS - Kepler rotators}
    \item{TGAS - Asteroseismic giants}
    \item{TGAS - Asteroseismic dwarfs/subgiants}
\end{itemize}

\section{The Method}

\begin{itemize}
    \item{Vert action (JZ) RMS as a function of age for various age
        indicators.}
        \subitem{ - Barnes (2007), (2010)/Mamajek \& Hillenbrand (2008)/ Angus
        (2015) /Matt (2012)/ van Saders (2013)/ van Saders (2016).}
    \item{Some understanding of how much we need to know RV to do this right.}
        \subitem{ - Simulate results with random RVs}
    \item{Use of this to cross-calibrate different age indicators.}
    \item{Some understanding of how sensitive we are to crazy selection.}
        \subitem{- Seismology, rotation, TGAS, Kepler.}
    \item{Comments on prospect for using JZ as an age indicator directly.}
        \subitem{- What precision?}
\end{itemize}

% Rotation period measurement. Citation for Angus et al? (Woohoo!)

% Red Giant relations.

% Model comparison.

\section{Results and Discussion}

% Findings

% Implications.

\section{Conclusion}

% We're awesome

% acknowledgements
This research was funded by the Simons Foundation.
Some of the data presented in this paper were obtained from the Mikulski
Archive for Space Telescopes (MAST).
STScI is operated by the Association of Universities for Research in
Astronomy, Inc., under NASA contract NAS5-26555.
Support for MAST for non-HST data is provided by the NASA Office of Space
Science via grant NNX09AF08G and by other grants and contracts.
This paper includes data collected by the Kepler mission. Funding for the
Kepler mission is provided by the NASA Science Mission directorate.

\bibliographystyle{plainnat}
\bibliography{granola}
\end{document}
